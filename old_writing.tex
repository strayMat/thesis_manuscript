
% RCT and Real world data
\paragraph{Recent surge in the collection of healtchare data, are drawing
    increasing attention to observational studies based on nonrandomized data
    sources for estimating causal effects.} Often referred as Real World Data
(RWD), they come from routine care Information System routine such as
Electronic Health Records (EHR) or insurance claims
\citep{wang2023emulation}. RWD are interesting opportunities to research
effectiveness of treatments in routine care data, generalize RCT findings to
a broader population, to explore treatment heterogeneity on subgroups
\citep{mant1999can, desai2021broadening}, and to gain first insights at
smaller costs than RCTs \citep{black1996we,bosdriesz2020evidence}. Moreover,
this data is well suited to measure the impact of health policies or changes
in professional guidelines. These opportunities have been well identified by
HTA agencies such as the \textit{Food and Drug Administration}
\citep{desai2021broadening,fda_real-world_2021}, the \textit{European
    Medicines Agency} \citep{flynn_marketing_2022}, the French \textit{Haute
    Autorité de Santé} \citep{vanier2023rapid} and the British \textit{National
    Institute for Health and Care Excellence} \citep{kent_nice_2022}.

%
Randomized Control Trials (RCT) have
long been the gold standard to measure whether the benefits of an intervention
outweigh its harms \citep{brook1986method}. However, routinely collected data
such as Electronic Health Records (EHR) or insurance claims
\citep{wang2023early} can also be leveraged to estimate treatment effects via
counterfactual prediction analyses \citep{hernan2019second}.  They are
particularly useful to study comparative effectiveness of treatments,
generalize RCT findings to broader populations, to explore treatment
heterogeneity on subgroups \citep{mant1999can,desai2021broadening}, and to
gain first insights at smaller costs than RCTs
\citep{black1996we,bosdriesz2020evidence}. However, the lack of controlled
interventions and the time-varying nature of such data open many threats to
the validity of causal inference \citep{bouvier2023should}.
Despite the interest in RWD for observational studies, a clear analysis
framework is required to avoid the inherent pitfalls of such studies, limit the
introduction of biases and derive insightful evidences \citep{hernan2019second,
    wang2023emulation}. It has been shown that failure to follow closely a ground
truth emulated trial,
yields substantially different results in observational studies: eg. for
cardiovascualar outcomes of antidiabetic or antiplatelet medications based on
insurance claims data \citep{schneeweiss2021conducting}. Comparing and
replicating real-world evidence studies for focused on the risk of major
bleeding
with dabigatran compared with warfarin, \cite{wang2022understanding} found that
60 to 88\% of variation remained unexplained. On the contrary, when closely
following the RCTs protocol, real-world evidence studies has been shown to be
highly consistent with RCTs \citep{wang2023emulation}.


% 


\subsection{Data for primary or secondary usages?}%
\label{subsec:cdw:data_usages}%

Uses covered by CDWs are not well characterized.

\paragraph{Primary use of health data focus on patient care.}
EHRs are primarily used to record patients' health state and care
\citep{safran_toward_2007,eu_primary_2022}. This information might be shared within the
healthcare team, but always with the goal to cary for the patient whose data is
collected.


\paragraph{Secondary use do not directly concern patient care.} Data can be
reused for other purposes, such as research, activity indicators for the
management or quality of care, optimization of the coding of medical
information, and feasibility studies
\citep{safran_toward_2007,cnil_referentiel_2021}.

\paragraph{Towards mix usages?}
Emerging paradigms such as personalized medicine encompass both types of use,
as in the case of learning systems that give rise to diagnosis, prognosis or
decision-making tools. During the \textit{learning phase}, data from a set of
patients is reused --secondary use-- to create the knowledge required to
operate the tool. In production, the tool takes input data from a specific
patient to improve his or her care --primary use.
