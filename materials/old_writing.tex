%%%%%%%%%%%%%%%%%%%%%%%%%%%%%%%%%%%%%%%%%%%%%%%%%%%%%%%%%%%%%%%%%%%%%%%%%%%%%%%%
% Résumé de la thèse en Français pour le grand public

Les bases de données de santé dites de vie réelle sont de plus en plus
nombreuses, exhaustives, avec des détails temporels précis. Contrairement aux
données utilisées dans la recherche clinique traditionnelle, elles capturent
l'organisation routinière des soins. Ces données de soins quotidiens ouvrent la
porte à de nouvelles questions de recherche, notamment en ce qui concerne la
qualité des soins, l'efficacité des interventions après leur mise sur le marché,
l'hétérogénéité de leurs bénéfices dans les populations mal desservies ou le
développement de traitements personnalisés. D'un autre côté, la complexité et la
nature à grande échelle de ces bases de données posent un certain nombre de
défis pour leur utilisation efficace. Pour remédier à ces problèmes, les
économètres et les épidémiologistes ont récemment proposé l'utilisation de
modèles d'estimation d'effets de traitement flexibles combinant l'inférence
causale et l'apprentissage automatique en grande dimension.

Dans un premier temps, nous illustrons par trois exemples la tension actuelle
entre ces nouvelles sources de données, l'apprentissage automatique et des
problématiques modernes de santé publique. Ces exemples motivent notre principale question
de recherche : Comment des modèles flexibles peuvent-ils aider à
fournir un traitement approprié à chaque patient afin d'améliorer sa santé ?%
Afin de mieux comprendre les infrastructures modernes de collecte
et d'analyse des dossiers patients informatisés (DPI), nous faisons la
synthèse d'entretiens semi-structurés menés dans le cadre d'une étude de cas
nationale portant sur les entrepôts de données cliniques des 32 hôpitaux
régionaux et universitaires français.%
Reconnaissant la difficulté d'accéder à des échantillons de grande taille et à
la puissance de calcul pour développer des modèles prédictifs performants, nous
étudions un gradient de complexité de différents systèmes prédictifs sur DPI.%
En se tournant vers le cadre causal, nous détaillons ensuite les éléments clés
nécessaires pour estimer de manière robuste l'effet du traitement à partir de
données de DPI variant dans le temps. Nous documentons l'impact de différents
choix méthodologiques pour l'étude de l'effet de l'albumine sur la mortalité
dans des cas de septicémie dans une base de soins intensifs.%
Dans des problèmes à grande dimension, le problème de sélection d'un modèle
causal performant parmi plusieurs possibilités devient essentiel. Sur une
simulation et trois ensembles de données semi-simulées, nous montrons que la
procédure du R risk, quoique plus complexe est la plus adaptée.


%%%%%%%%%%%%%%%%%%%%%%%%%%%%%%%%%%%%%%%%%%%%%%%%%%%%%%%%%%%%%%%%%%%%%%%%%%%%%%%%


% RCT and Real world data
\paragraph{Recent surge in the collection of healtchare data, are drawing
    increasing attention to observational studies based on nonrandomized data
    sources for estimating causal effects.} Often referred as Real World Data
(RWD), they come from routine care Information System routine such as
Electronic Health Records (EHR) or insurance claims
\citep{wang2023emulation}. RWD are interesting opportunities to research
effectiveness of treatments in routine care data, generalize RCT findings to
a broader population, to explore treatment heterogeneity on subgroups
\citep{mant1999can, desai2021broadening}, and to gain first insights at
smaller costs than RCTs \citep{black1996we,bosdriesz2020evidence}. Moreover,
this data is well suited to measure the impact of health policies or changes
in professional guidelines. These opportunities have been well identified by
HTA agencies such as the \textit{Food and Drug Administration}
\citep{desai2021broadening,fda_real-world_2021}, the \textit{European
    Medicines Agency} \citep{flynn_marketing_2022}, the French \textit{Haute
    Autorité de Santé} \citep{vanier2023rapid} and the British \textit{National
    Institute for Health and Care Excellence} \citep{kent_nice_2022}.

%
Randomized Control Trials (RCT) have
long been the gold standard to measure whether the benefits of an intervention
outweigh its harms \citep{brook1986method}. However, routinely collected data
such as Electronic Health Records (EHR) or insurance claims
\citep{wang2023early} can also be leveraged to estimate treatment effects via
counterfactual prediction analyses \citep{hernan2019second}.  They are
particularly useful to study comparative effectiveness of treatments,
generalize RCT findings to broader populations, to explore treatment
heterogeneity on subgroups \citep{mant1999can,desai2021broadening}, and to
gain first insights at smaller costs than RCTs
\citep{black1996we,bosdriesz2020evidence}. However, the lack of controlled
interventions and the time-varying nature of such data open many threats to
the validity of causal inference \citep{bouvier2023should}.
Despite the interest in RWD for observational studies, a clear analysis
framework is required to avoid the inherent pitfalls of such studies, limit the
introduction of biases and derive insightful evidences \citep{hernan2019second,
    wang2023emulation}. It has been shown that failure to follow closely a ground
truth emulated trial,
yields substantially different results in observational studies: eg. for
cardiovascualar outcomes of antidiabetic or antiplatelet medications based on
insurance claims data \citep{schneeweiss2021conducting}. Comparing and
replicating real-world evidence studies for focused on the risk of major
bleeding
with dabigatran compared with warfarin, \cite{wang2022understanding} found that
60 to 88\% of variation remained unexplained. On the contrary, when closely
following the RCTs protocol, real-world evidence studies has been shown to be
highly consistent with RCTs \citep{wang2023emulation}.


% 


\subsection{Data for primary or secondary usages?}%
\label{subsec:cdw:data_usages}%

Uses covered by CDWs are not well characterized.

\paragraph{Primary use of health data focus on patient care.}
EHRs are primarily used to record patients' health state and care
\citep{safran_toward_2007,eu_primary_2022}. This information might be shared within the
healthcare team, but always with the goal to cary for the patient whose data is
collected.


\paragraph{Secondary use do not directly concern patient care.} Data can be
reused for other purposes, such as research, activity indicators for the
management or quality of care, optimization of the coding of medical
information, and feasibility studies
\citep{safran_toward_2007,cnil_referentiel_2021}.

\paragraph{Towards mix usages?}
Emerging paradigms such as personalized medicine encompass both types of use,
as in the case of learning systems that give rise to diagnosis, prognosis or
decision-making tools. During the \textit{learning phase}, data from a set of
patients is reused --secondary use-- to create the knowledge required to
operate the tool. In production, the tool takes input data from a specific
patient to improve his or her care --primary use.


\subsection{Origin of causality in epidemiology}%

The Environment and Disease: Association or Causation? by Sir Austin Bradford Hill

He narrows the problem to the association between two variables: exposure and
outcome. He then list nine viewpoints to consider when assessing the evidence of
a causal relationship between a supposed cause and an observed effect.

\begin{enumerate}
    \item Strength of association
    \item Consistency: has it been repeatedly observed by different persons, in
          different places, circumstances and times? It is linked to external
          validity.
    \item Specificity: to the body sites involved, to the disease, to the
          population.
    \item Temporality
    \item Biological gradient ie. dose-response relationship
    \item Plausibility: we could not demand this criteria as plausibility is only
          linked to our biological knowledge at a given time.
    \item Coherence with the existing body of knowledge.
    \item Experiment: the possibility to modify the supposed cause and observe
          the effect.
    \item Analogy: the existence of similar relationships.
\end{enumerate}


He think that multi-causality is more frequent, but if we knew all the causess,
we might get back to a single factor.

*Like fire, the X2 test is an excellent servant and a bad master.*