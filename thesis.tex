\documentclass{report}


\usepackage{arxiv}

\usepackage[utf8]{inputenc} % allow utf-8 input
\usepackage[T1]{fontenc}    % use 8-bit T1 fonts
\usepackage{hyperref}       % hyperlinks
\usepackage{url}            % simple URL typesetting
\usepackage{booktabs}       % professional-quality tables
\usepackage{makecell}
\usepackage{amsfonts}       % blackboard math symbols
\usepackage{nicefrac}       % compact symbols for 1/2, etc.
\usepackage{microtype}      % microtypography
\usepackage{lipsum}
\usepackage{graphicx}
\usepackage[dvipsnames]{xcolor}
\usepackage[natbib=true,style=authoryear,maxnames=999,maxcitenames=2]{biblatex}
\usepackage{tabularx}
\usepackage{multirow}
\usepackage{pifont}
\usepackage{csvsimple}

\usepackage{todonotes}
\newcommand{\md}[1]{\todo[color=blue!40,inline]{Matt: #1}}
\newcommand{\ts}[1]{\todo[color=orange!40,inline]{Tristan: #1}}


% color commands
\colorlet{P}{ForestGreen}
\colorlet{I}{MidnightBlue}
\colorlet{C}{YellowOrange}
\colorlet{O}{DarkOrchid}

% equations commands
\newcommand{\indep}{\perp \!\!\! \perp}
\newtheorem{assumption}{Assumption}


\bibliography{references} 


\title{
  Representations and inference from time-varying routine care data 
}

\author{
  Matthieu Doutreligne \\
  %% Address \\
  %% \texttt{email} \\
  %% \And
  %% Coauthor \\
  %% Affiliation \\
  %% Address \\
  %% \texttt{email} \\
  %% \And
  %% Coauthor \\
  %% Affiliation \\
  %% Address \\
  %% \texttt{email} \\
}

\begin{document}
\maketitle


\begin{abstract}

  Real World Databases are increasingly accessible, exhaustive and with
  fine temporal details. Unlike traditional data used in clinical research, they
  describe the routine organization of care. These day-to-day records of
  patients care enable new research questions, notably concerning the efficiency
  of interventions after market access, the heterogeneity of their benefits in
  under-served populations or the development of personalized medicine. On the
  other hand, the complexity and large-scale nature of these databases pose a
  number of challenges for effectively answering these questions. To remedy
  these problems, econometricians and epidemiologists have recently proposed
  the use of flexible models combining causal inference with high-dimensional
  machine learning.

  Chapter 1 uses a national case study of the 32 French regional and
  university hospitals to highlight key aspects of modern Clinical Data
  Warehouses (CDWs), the cornerstone infrastructure of precision medicine. From
  semi-structured interviews and an analysis of declared observational studies
  on CDWs in France, I highlight both the current potential and challenges of
  leveraging these routinely collected data for research purposes.

  Acknowledging the difficulty to access large sample sizes and computational
  power to develop generalizable predictive models, Chapter 2 leverages
  distributional representations of medical concepts for clinical tasks. I show
  that these sharable representations out-perform more complex models for
  small sample datasets.

  Despite a current focus on artificial intelligence, healthcare does not seek
  purely predictive models, but appropriate interventions that will benefit
  specific patients. This is a decision making problem highly amenable to
  counterfactual prediction rather than statistical learning. Chapter 3 emulates
  a clinical trial to evaluate an intervention in the intensive care unit. I
  highlight the importance of the different practical choices when developing
  counterfactual prediction algorithms on time-varying data collected in routine
  care.

  In high-dimensional settings such as time-varying data and heterogeneity of
  interventions on subgroups, the selection of hyper-parameters for the causal
  model is crucial to avoid under- or over-learning. Chapter 4 demonstrates the
  fragility of the usual evaluation metrics (Mean Squared Error) for
  counterfactual predictions. I highlight the performance of the doubly robust
  R-risk over other existing risks and discuss their effectiveness regimes in a
  simulation and three semi-simulated datasets.

  Chapter 5 concludes by highlighting the potential of combining machine
  learning methods and routine care data to shed light on current public health
  issues. I discuss new avenues to improve the development and the evaluation of
  tailored interventions, public health policies or quality-of-care indicators.

\end{abstract}


% keywords can be removed
%\keywords{First keyword \and Second keyword \and More}


\chapter{Introduction}

\section{Context}

% RCT and Real world data
\paragraph{Recent surge in the collection of healtchare data, are drawing
  increasing attention to observational studies based on nonrandomized data
  sources for estimating causal effects.} Often referred as Real World Data
(RWD), they come from routine care Information System routine such as
Electronic Health Records (EHR) or insurance claims \citep{wang2023emulation}.
RWD are interesting opportunities to research effectiveness of treatments in
routine care data, generalize RCT findings to a broader population, to explore
treatment heterogeneity on subgroups \citep{mant1999can, desai2021broadening},
and to gain first insights at smaller costs than RCTs
\citep{black1996we,bosdriesz2020evidence}. Moreover, this data is well suited
to measure the impact of health policies or changes in professional
guidelines. These opportunities have been well identified by HTA agencies such
as the \textit{Food and Drug Administration}
\citep{desai2021broadening,fda_real-world_2021}, the \textit{European
  Medicines Agency} \citep{flynn_marketing_2022}, the French \textit{Haute
  Autorité de Santé} \citep{vanier2023rapid} and the British \textit{National
  Institute for
  Health and Care Excellence} \citep{kent_nice_2022}.


Randomized Control Trials (RCT) have
long been the gold standard to measure whether the benefits of an intervention
outweigh its harms \citep{brook1986method}. However, routinely collected data
such as Electronic Health Records (EHR) or insurance claims
\citep{wang2023early} can also be leveraged to estimate treatment effects via
counterfactual prediction analyses \citep{hernan2019second}.  They are
particularly useful to study comparative effectiveness of treatments,
generalize RCT findings to broader populations, to explore treatment
heterogeneity on subgroups \citep{mant1999can,desai2021broadening}, and to
gain first insights at smaller costs than RCTs
\citep{black1996we,bosdriesz2020evidence}. However, the lack of controlled
interventions and the time-varying nature of such data open many threats to
the validity of causal inference \citep{bouvier2023should}.


% ML in healthcare 
Machines read medical images faster
and more efficiently than most practitioners \citep{zhou2021review}.
Structured data from Electronic Health Records \citep{rajkomar2018scalable} or
administrative databases \citep{beaulieu2021machine} outperform rule-based
clinical scores in predicting patient's readmission, in-hospital mortality or
future comorbidities \citep{li2020behrt}. Recently, large language models
leveraged clinical notes from several hospitals for length of stay prediction
\citep{jiang2023health}. Hope is high that LLMs models will soon be able to
help practitioners during consultation \citep{lee2023benefits}.


\section{Motivation}

Despite the interest in RWD for observational studies, a clear analysis
framework is required to avoid the inherent pitfalls of such studies, limit the
introduction of biases and derive insightful evidences \citep{hernan2019second,
  wang2023emulation}. It has been shown that failure to follow closely a ground
truth emulated trial,
yields substantially different results in observational studies: eg. for
cardiovascualar outcomes of antidiabetic or antiplatelet medications based on
insurance claims data \citep{schneeweiss2021conducting}. Comparing and
replicating real-world evidence studies for focused on the risk of major
bleeding
with dabigatran compared with warfarin, \cite{wang2022understanding} found that
60 to 88\% of variation remained unexplained. On the contrary, when closely
following the RCTs protocol, real-world evidence studies has been shown to be
highly consistent with RCTs \citep{wang2023emulation}.
\clearpage

\section{Definitions}

\subsection{Interventional vs Observational}

\paragraph{Interventional data}

\textit{In an experiment, the reason for the exposure assignment is solely to
  suit the objectives of the study; if people receive their exposure assignment
  based on considerations other than the study protocol, it is not a true
  experiment} \citep{rothman2012epidemiology}

Reinforcement learning can be seen as a form if
interventional data collection where intervention probabilities are known
\cite{bareinboim2015bandits}


\chapter{Good practices for clinical data warehouse implementation: A case study
  in France}\label{chap:good_practices}


\section{Results}\label{sec:good_practices:results}
\subsection{Data}\label{subsec:good_practices:results:data}

\paragraph{Strong interest for CDW in the context of care}

Some CDWs develop specific applications that provide new functionalities
compared to care software. Search engines can be used to query all the
hospital's data gathered in the CDW, without data compartmentalization between
different softwares. Dedicated interfaces can then offer a unified view of the
history of a patient's data, with inter-specialty transversality, which is
particularly valuable in internal medicine. These cross-disciplinary search
tools also enable healthcare professionals to conduct rapid searches in all the
texts, for example, to find similar patients \citep{garcelon2017finding}.
%
There is a growing interest in such computational phenotyping tools to support
the development of digital health solutions \citep{wen2023impact}.
%
Uses for prevention, automation of repetitive tasks, and care coordination are
also highlighted. Concrete examples are the automatic sorting of hospital
prescriptions by order of complexity or the setting up of specialized channels
for primary or secondary prevention.


\chapter{Conclusion}


Predictive medecine is hard

Modern healthcare burdens and costs are driven by chronic diseases where death
is not the only outcome of interest.

Focus on smaller rewards, on which experiments are easier to conduct and where
error is possible: it would allow better learning for decision making since we
can repeat more experiments (limit: signals could be highly delayed).

Text is pervasive, we are using it to communicate and to log most of our
information. We should leverage it more in the context of care (limit:
temporality is hard to capture but is a key aspect of causal inference).

\printbibliography

\appendix

\clearpage

\section{Causal Diagrams}

\end{document}
