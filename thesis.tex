\documentclass{article}


\usepackage{arxiv}

\usepackage[utf8]{inputenc} % allow utf-8 input
\usepackage[T1]{fontenc}    % use 8-bit T1 fonts
\usepackage{hyperref}       % hyperlinks
\usepackage{url}            % simple URL typesetting
\usepackage{booktabs}       % professional-quality tables
\usepackage{makecell}
\usepackage{amsfonts}       % blackboard math symbols
\usepackage{nicefrac}       % compact symbols for 1/2, etc.
\usepackage{microtype}      % microtypography
\usepackage{lipsum}
\usepackage{graphicx}
\usepackage[dvipsnames]{xcolor}
\usepackage[natbib=true,style=authoryear,maxnames=999,maxcitenames=2]{biblatex}
\usepackage{tabularx}
\usepackage{multirow}
\usepackage{pifont}
\usepackage{csvsimple}

\usepackage{todonotes}
\newcommand{\md}[1]{\todo[color=blue!40,inline]{Matt: #1}}
\newcommand{\ts}[1]{\todo[color=orange!40,inline]{Tristan: #1}}


% color commands
\colorlet{P}{ForestGreen}
\colorlet{I}{MidnightBlue}
\colorlet{C}{YellowOrange}
\colorlet{O}{DarkOrchid}

% equations commands
\newcommand{\indep}{\perp \!\!\! \perp}
\newtheorem{assumption}{Assumption}


\bibliography{references} 


\title{What practical choices impact counterfactual predictions from EHR data ?}

\author{
  Matthieu Doutreligne \\
  %% Address \\
  %% \texttt{email} \\
  %% \And
  %% Coauthor \\
  %% Affiliation \\
  %% Address \\
  %% \texttt{email} \\
  %% \And
  %% Coauthor \\
  %% Affiliation \\
  %% Address \\
  %% \texttt{email} \\
}

\begin{document}
\maketitle
\begin{abstract}


  Recent advances in machine learning are bringing the promise of predictive
  medicine closer to reality. Machines read medical images faster and more
  efficiently than most practitioners \citep{zhou2021review}. Structured data from Electronic Health
  Records \citep{rajkomar2018scalable} or administrative databases
  \citep{beaulieu2021machine} outperform rule-based clinical scores in
  predicting patient's readmission, in-hospital mortality or future
  comorbidities \citep{li2020behrt}. Recently, large language models leveraged
  clinical notes from several hospitals for length of stay prediction
  \citep{jiang2023health}. Hope is high that LLMs models will soon be able to
  help practitioners during consultation \citep{lee2023benefits}.

  Randomized Control Trials (RCT) have
  long been the gold standard to measure whether the benefits of an intervention
  outweigh its harms \citep{brook1986method}. However, routinely collected data
  such as Electronic Health Records (EHR) or insurance claims
  \citep{wang2023early} can also be leveraged to estimate treatment effects via
  counterfactual prediction analyses \citep{hernan2019second}.  They are
  particularly useful to study comparative effectiveness of treatments,
  generalize RCT findings to broader populations, to explore treatment
  heterogeneity on subgroups \citep{mant1999can,desai2021broadening}, and to
  gain first insights at smaller costs than RCTs
  \citep{black1996we,bosdriesz2020evidence}. However, the lack of controlled
  interventions and the time-varying nature of such data open many threats to
  the validity of causal inference \citep{bouvier2023should}.


\end{abstract}


% keywords can be removed
%\keywords{First keyword \and Second keyword \and More}



\section{Introduction}

\subsection{Context}


\paragraph{Recent surge in the collection of healtchare data, are drawing
  increasing attention to observational studies based on nonrandomized data
  sources for estimating causal effects.} Often referred as Real World Data
(RWD), they come from routine care Information System routine such as
Electronic Health Records (EHR) or insurance claims \citep{wang2023emulation}.
RWD are interesting opportunities to research effectiveness of treatments in
routine care data, generalize RCT findings to a broader population, to explore
treatment heterogeneity on subgroups \citep{mant1999can, desai2021broadening},
and to gain first insights at smaller costs than RCTs
\citep{black1996we,bosdriesz2020evidence}. Moreover, this data is well suited
to measure the impact of health policies or changes in professional
guidelines. These opportunities have been well identified by HTA agencies such
as the \textit{Food and Drug Administration}
\citep{desai2021broadening,fda_real-world_2021}, the \textit{European
  Medicines Agency} \citep{flynn_marketing_2022}, the French \textit{Haute
  Autorité de Santé} \citep{vanier2023rapid} and the British \textit{National
Institute for
  Health and Care Excellence} \citep{kent_nice_2022}.

\subsection{Motivation}

Despite the interest in RWD for observational studies, a clear analysis
framework is required to avoid the inherent pitfalls of such studies, limit the
introduction of biases and derive insightful evidences \citep{hernan2019second,
  wang2023emulation}. It has been shown that failure to follow closely a ground
truth emulated trial,
yields substantially different results in observational studies: eg. for
cardiovascualar outcomes of antidiabetic or antiplatelet medications based on
insurance claims data \citep{schneeweiss2021conducting}. Comparing and
replicating real-world evidence studies for focused on the risk of major
bleeding
with dabigatran compared with warfarin, \cite{wang2022understanding} found that
60 to 88\% of variation remained unexplained. On the contrary, when closely
following the RCTs protocol, real-world evidence studies has been shown to be
highly consistent with RCTs \citep{wang2023emulation}.
\clearpage

\printbibliography

\appendix

\clearpage

\section{Causal Diagrams}

\end{document}
