\documentclass{report}


\usepackage{arxiv}

\usepackage[utf8]{inputenc} % allow utf-8 input
\usepackage[T1]{fontenc}    % use 8-bit T1 fonts
\usepackage{hyperref}       % hyperlinks
\usepackage{url}            % simple URL typesetting
\usepackage{booktabs}       % professional-quality tables
\usepackage{makecell}
\usepackage{amsfonts}       % blackboard math symbols
\usepackage{nicefrac}       % compact symbols for 1/2, etc.
\usepackage{microtype}      % microtypography
\usepackage{lipsum}
\usepackage{graphicx}
\usepackage[natbib=true,style=authoryear,maxnames=999,maxcitenames=2]{biblatex}
\usepackage{tabularx}
\usepackage{multirow}
\usepackage{pifont}
\usepackage{csvsimple}

% notes in the margin
\usepackage{todonotes}
\usepackage{snaptodo}

\setlength{\marginparwidth}{1.4cm}
\setlength{\marginparsep}{-.01cm}
%\setlength\textwidth{177.8mm}

\snaptodoset{block rise=1em}
\snaptodoset{margin block/.style={font=\tiny}} \newcommand{\gv}[1]{%
    \snaptodo[margin block/.append style=green!50!black]{%
	\sloppy\textbf{Gael}: #1}}
\snaptodoset{margin block/.style={font=\tiny}} \newcommand{\md}[1]{%
    \snaptodo[margin block/.append style=blue]{%
	\sloppy\textbf{Matthieu}: #1}}
  
% color commands
\colorlet{P}{ForestGreen}
\colorlet{I}{MidnightBlue}
\colorlet{C}{YellowOrange}
\colorlet{O}{DarkOrchid}

% equations commands
\newcommand{\indep}{\perp \!\!\! \perp}
\newtheorem{assumption}{Assumption}


\bibliography{references} 


\title{
  Representations and inference from time-varying routine care data 
}

\author{Matthieu Doutreligne}
\date{}

\begin{document}
\maketitle

\begin{abstract}

  Real World Databases are increasingly accessible, exhaustive and with
  fine temporal details. Unlike traditional data used in clinical research, they
  describe the routine organization of care. These day-to-day records of
  patients care enable new research questions, notably concerning the efficiency
  of interventions after market access, the heterogeneity of their benefits in
  under-served populations or the development of personalized medicine. On the
  other hand, the complexity and large-scale nature of these databases pose a
  number of challenges for effectively answering these questions. To remedy
  these problems, econometricians and epidemiologists have recently proposed
  the use of flexible models combining causal inference with high-dimensional
  machine learning.

  Chapter 1 uses a national case study of the 32 French regional and
  university hospitals to highlight key aspects of modern Clinical Data
  Warehouses (CDWs), the cornerstone infrastructure of precision medicine. From
  semi-structured interviews and an analysis of declared observational studies
  on CDWs in France, I highlight both the current potential and challenges of
  leveraging these routinely collected data for research purposes.

  Acknowledging the difficulty to access large sample sizes and computational
  power to develop generalizable predictive models, Chapter 2 leverages
  distributional representations of medical concepts for clinical tasks. I show
  that these sharable representations out-perform more complex models for
  small sample datasets.

  Despite a current focus on artificial intelligence, healthcare does not seek
  purely predictive models, but appropriate interventions that will benefit
  specific patients. This is a decision making problem highly amenable to
  counterfactual prediction rather than statistical learning. Chapter 3 emulates
  a clinical trial to evaluate an intervention in the intensive care unit. I
  highlight the importance of the different practical choices when developing
  counterfactual prediction algorithms on time-varying data collected in routine
  care.

  In high-dimensional settings such as time-varying data and heterogeneity of
  interventions on subgroups, the selection of hyper-parameters for the causal
  model is crucial to avoid under- or over-learning. Chapter 4 demonstrates the
  fragility of the usual evaluation metrics (Mean Squared Error) for
  counterfactual predictions. I highlight the performance of the doubly robust
  R-risk over other existing risks and discuss their effectiveness regimes in a
  simulation and three semi-simulated datasets.

  Chapter 5 concludes by highlighting the potential of combining machine
  learning methods and routine care data to shed light on current public health
  issues. I discuss new avenues to improve the development and the evaluation of
  tailored interventions, public health policies or quality-of-care indicators.

\end{abstract}
% keywords can be removed
%\keywords{First keyword \and Second keyword \and More}

\tableofcontents




\chapter{Introduction: amazing opportunities of health data?}\label{chap:intro}
\section{How I came into this landscape}\label{sec:intro:landscape}

I will present my progression among these opportunities.

\subsection{Context: From a modern statistical formation to the first contact with epidemiology}\label{subsec:intro:context}

\paragraph{Learning statistics during the Natural Language Processing revolution}

Machine learning techniques, a subfield of artificial intelligence, became
increasingly good as solving complex tasks where previous pattern recognition
techniques were struggling. First, categorized into supervised and unsupervised
approaches, the pre-training paradigm became increasingly popular in the 2010's:
\cite{halevy2009unreasonable} proposed to take advantages of regularities
present in large piles of data to design automatically interesting features for
large application domains. The practical successes came in image processing
\citep{}, then Natural Language Processing \citep{} and structural biology.
%

\paragraph{APHP: Removing identification elements from clinical notes}

Continuous improvements in natural language processing led to the intuition that
soon the information contained in clinical texts could be mined by appropriate
to serve clinical research.

However, privacy concern: the access to very detailed elements of patient cares
is no more restricted to the medical staff.

Need for pseudonymization : use newly to reach State of the Art pseudonymization
\citep{dernoncourt2017identification, paris2019desidentification}.

\paragraph{Billing claims: an underused huge pile of data}

French ministry department of statistics had billions of claims extracted in a
dedicated server. But the amount of data made the job impossible.

I learned that distributed computing (aka appropriate tools), summary statistics
and good documentation.

I began to wonder what were the appropriate methodological tools to extract
relevant information from this data. Hammer that sees a nail : Can I use language
modeling techniques to mine this data ?

First questions: how useful are such wealth of data ? Is unsupervised learning
useful ? Is prediction a useful goal ? Ironically, at the end of the day, I
replicated unsupervised learning results from \citep{beam2019clinical} but would
not know how to use it. I moved towards more interpretable dictionary learning
with control groups analyses to study polymedication.

% Not sure it is useful
Covid-19: National healthcare monitoring during the epidemics. Integrated data
pipeline are very useful during crisis for monitoring: If you do not see all the
data creating process, standard pipelines, standard tests and documentation : it
is inherently a group effort.

\subsection{Wrap-up: Increasing data collection and computing power}

\paragraph{Data is collected at massive scale in healthcare} centralized
into dedicated databases for analysis: eg. APHP clinical notes, French.

\paragraph{Models are scaling up}
A whole generation of quantitative researchers has been trained with flexible
data-hungry models. The unreasonable effectiveness of data \citep{halevy2009unreasonable}.

A recent work on the classification of data and models (AI) in clinical medicine
suggests three main usages: AI in clinical practice, clinical research on AI/ML
device and applications and AI/ML used to conduct clinical research
\citep{haug2023artificial}.

\subsection{What pressing needs to use health data}

Shortliffe, the inventer of MYCINE \citep{shortliffe1974mycin}, the first
rule-based artificial intelligence expert systems judges \textit{medical
  practice, and biomedical research, are inherently information-management tasks}
\citep{patel2009coming}.

However, the incentives to collect and analyze data are not aligned between
healthcare actors.

\paragraph{Healthcare practice}

Automation of tedious tasks, decision support systems

\paragraph{Research}

Drug development, evidence based medicine, biomarkers, epidemiology ie.
understanding of the mechanisms of disease \textit{the study of the distribution
  and determinants of disease frequency} \citep{macmahon1970epidemiology}

\paragraph{Public policies: monitoring and evaluation}

Guidelines, quality of care, public health
Health Technology Agencies interest: organizational impact of health products,
potential for real world efficiency (entire life cycle of the product), quality,
security, pertinence and security of care, health monitoring, health management.

% I am not totally naive
\paragraph{Marketing}

Tailored recommendations, market access

\section{The data: Electronic Health Records}\label{sec:intro:data}

\subsection{Various types of data: Real World Data}\label{subsec:intro:real_world_data}

Real world data refer to routinely collected data

Gradient and distinction between research data and opportunistic data
collection.

\paragraph{Traditional research data collections}

Hand made data collection for one research question with a specific protocol,
Specialized registry and cohorts


\paragraph{Claims}

Billing system, eg. PMSI.

Advantages (space and time coverage, scale, structure), disadvantages (not clinic, no exam results, few measures, heterogeneity of collection)

\paragraph{EHRs and Hospital Information System}

Increasing informatization
EHR at the center
Other applications part of HIS: give examples.

\paragraph{Clinical Data Warehouse} An infrastructure
is needed to pool data from one or more medical information systems. Definition
of Clinical Data Warehouse: Health data warehouses (HDWs) refer to the pooling of
data from one or more medical information systems, in a homogeneous format for
reuse in management, research or care.


The 4 phases of data flow from the various sources that make up the HIS (figure)
Focus of this thesis: Clinical Data Warehouse and EHR, though most of my work should apply to claims



\subsection{Interventional data vs observational
  data}\label{subsec:intro:interventional_vs_observational}


Interventional data contains interaction with the patients / environment where
the intervention probabilities are known. Eg. RCT : fixed probability for any
patient (statistical unit): first RCT, RCTs as the basis of Evidence Based
Medicine in the late 80s.

\textit{In an experiment, the reason for the exposure assignment is solely to
  suit the objectives of the study; if people receive their exposure assignment
  based on considerations other than the study protocol, it is not a true
  experiment} \citep{rothman2012epidemiology}

Pbs of external validity raised in economics (Deaton,
cf intro de la revue de Bénédicte). Focus is such problems is probably greater
in economics because situations are not well controlled at all: very far from
labs settings. Clinical situation is closer to lab (clinical epidemiology !=
social epidemiology) What about medico-economics. I do not address this question
in the present thesis, but it is a clear motivation

%
Observational data cannot intervene in any way with the patients. It should rely
on observation alone to estimate intervention probability.

% Interventional vs Observational: A gradient as well 
Conditional probabilities are interventional as well: conditionally randomized
experiments in epidemiology \citep{hernan2020causal}. This concept is known as
reinforcement learning in the Machine Learning community (bareinboim2015bandits)
where intervention probabilities are known
\cite{bareinboim2015bandits}.

\subsection{Focus of this thesis: real world data, observational}\label{subsec:intro:focus_data}

Inconvenient of interventional data : Increasing costs of RCTs,

Today's public health issues are closely linked to routine care, not ideal care:
resource constraint, chronic disease. Not so new as it was already major issues
mentioned in the late 60s \cite{rutstein1967coming}: \textit{1) modern
  medicine's skyrocketing costs; 2) the chaos of an information explosion
  involving both paperwork pro- liferation and large amounts of new knowledge that
  no single physician could hope to digest; 3) a geographic maldistribution of
  MD's; 4) increasing demands on the physician's time as increasing numbers of
  individuals began to demand quality medical care.}

Rise of digitalization and computing power suggesting new research methods

\section{Two cultures of statistics for health}\label{sec:intro:two_cultures}

In 2001, \cite{breiman2001statistical} clearly separates two statistical
cultures: a predominant community at the time focused on models, another one
relying on predictive accuracy.

Epidemiological statistics tends to be closer to the first model-based culture
whereas artificial intelligence in medicine embraced the second one.

\begin{baground_box_left}
  \subsection{Machine Learning Framework}\label{subsec:intro:machine_learning}

  Departing from carefully designing functional forms for statistical models,
  seeking algorithm efficiency and out-of sample predictive accuracy.

  \paragraph{Rashomon effect: multiple models can equally well model a given dataset}
  % ref to appendix to describe ML algorithms: at least ridge, random forest, boosting.

  \paragraph{Model selection}
  Hyper-parameters selection, cross-validation (stone, 1974) and Textbook figure

  \subsection{Biostatistics Framework}\label{subsec:intro:biostatistics_framework}

  In medical journals, Cox model for survival data and logistic regression for
  binary outcomes have been the standard for publication.

  Focus on variable selection: root in hand collected variables in carefully
  designed subpopulation ?


\end{baground_box_left}

\subsection{One choice of perspective: recent statistical
  learning}\label{subsec:intro:recent_statistical_learning}
% 
\paragraph{This perspective reflects more naturally my progression:} Learning
statistics in the age of deep learning pushes towards adopting a view on
statistics that consider mechanisms as generally unknown, and predictive
accuracy as the hallmark of validity \citep{breiman2001statistical}.
% 
% ML in healthcare 
\paragraph{Recent discourse and successes are heavily influenced by this line of
  though:} Machines read medical images faster and more efficiently than most
practitioners \citep{zhou2021review}. Structured data from Electronic Health
Records \citep{rajkomar2018scalable} or administrative databases
\citep{beaulieu2021machine} outperform rule-based clinical scores in
predicting patient's readmission, in-hospital mortality or future
comorbidities \citep{li2020behrt}. Recently, large language models (LLMs)
leveraged clinical notes from several hospitals for length of stay prediction
\citep{jiang2023health}. Hope is high that LLMs models will soon be able to
help practitioners during consultation \citep{lee2023benefits}.


\section{Notions of causality}\label{sec:intro:causality}

\begin{baground_box_left}

  \subsection{Association is not causation}\label{subsec:intro:causation}

  \paragraph{Observational data can have different causal interpretations}

  One statistical model yields multiple causal models, only one of which
  correctly reflects the reality : find a good medical example (otherwise look
  into Peter jonas's book or \cite[chapter~36]{murphy2022probabilistic}).

  \paragraph{Ladder of causality}

  Distinction between association and causation (ladder of causation): first
  insight that different tools than statistics are needed.

  \paragraph{Causality in epidemiology}

  Opening an epidemiological book, confounding is the very first explained concept
  \citep{rothman2012epidemiology}.

  Causes are mainly presented as binary variables absent or present in \citep{}.
  Their strengh is only defined at a population level, making them relative to a
  given population \citep[chapter~3]{rothman2012epidemiology}. This point of
  view is rather aligned with statistical modeling related to variable
  significance, not with model performances.

  Interesting concept of promoter --the last component cause-- that

  Importance of validity by generalization of theory rather than statistical
  generalization: established epidemiological knowledge should \textit{tell us
    what to expect in people or settings that were not studied.} Statistical
  generalization is considered by \cite{rothman2012epidemiology} as applied
  epidemiology, focused on specific cases and not as the core of epidemiological
  science.

  \paragraph{Causality in Machine Learning}

  Relation to dataset shift \citep{subbaswamy2020development}

  \subsection{Neyman-rubin causal framework}\label{subsec:intro:causal_framework}

  - PO notations
  - The rest will be presented in chapter 4 and 5.
  - Send back to appendix the DM (in randomized experiments)

\end{baground_box_left}

\section{Overview and contributions}\label{sec:intro:contributions}



\chapter{Potential and challenges of Clinical Data Warehouse, a case study in France}\label{chapter:cdw}

\section{Abstract}\label{sec:cdw:abstract}
\section{Motivation and background: a changing world}\label{sec:cdw:motivation}

\subsection{From primary to secondary data usages}\label{subsec:cdw:data_usages}

\paragraph{Primary data usages}
Primary usages directly serve one patient care.

\paragraph{Secondary data usages}
Secondary usages do not concern directly the care and support of one patient: research, quality or management indicators, billings.

\paragraph{Towards mix usages: the learning health system paradigm}

The notion of learning health system \citep{mcginnis2013best}.

Mix usages such as learning algorithms

\subsection{Healthcare data collection is heavily influenced from the local healthcare organization}

\paragraph{Centralized data collections: ex. Israël, HIRA}

\paragraph{More heterogeneous data collections structured into networks} ex. from

US, Korea, Germany,

\paragraph{The case of France}

centralized national insurer but scattered hospital ehrs. Projects in other areas: eg. cancer (unicancer), gp (cnge project, darmoni). National initiative to develop and structure hospital cdws.

\section{Speaking to the data collectors: interviews of French
  University Hospitals}\label{sec:cdw:methods}

\subsection{Interviews and study coverage}\label{subsec:cdw:interviews}

\paragraph{Semi-structured interviews}

Topics, questions, link to full data and questionnaires in appendix.

\paragraph{Regional and university hospitals in France: different levels of maturity}

Scope of 32 CHUs, out of the 3000 care sites in France to yield exhaustive
conclusion on a restricted scope. Drive most specialized care, research in their
core mission. Date of interview

\paragraph{Focus on the 18 CDWs with highest level of maturity}

The denominator for the quantitative results is the 18 CDWs in production

\subsection{A classification of observational
  studies}\label{subsec:cdw:classification}

Contrast with classical epidemiology study types and the notion of experiment \citep{rothman2012epidemiology}.

\paragraph{Outcome frequency}
\paragraph{Population characterization}
\paragraph{Risk factors}
\paragraph{Treatment effect}
\paragraph{Development of diagnostic or prognostic algorithms}
\paragraph{Medical informatics}


\section{Observations from a rapidly evolving and heterogeneous
  ecosystem}\label{sec:cdw:results}

\subsection{Governance: CDW are federating multiple teams in the
  hospital}\label{subsec:cdw:results:governance}
\paragraph{Initiation and actors}
Figure temporality of CDW
Federating potential
Multiple departments involved
Multiple skills involved, strong ties with the academics
In-house solution development vs industrialization

\paragraph{Management of studies}

Scientific committee and project follow-up platform

\subsection{Uneven transparency of ongoing
  studies}\label{subsec:cdw:results:transparency} Uneven public reference on
hospital websites of ongoing studies.

\subsection{Triple usage of data: research, management,
  clinic}\label{subsec:cdw:results:data}

\paragraph{Strong dependance to the HIS}

Data collected reflect data collection
Exemples, AP-HP, HCLs

\paragraph{Categories of Data}

Main functionalities of HIS are the same. Common base Details: Table Takeaway:
most of the current accessible data are billing, administrative and text,
importantly, low access to temporality.

\paragraph{Data reuse: research}
Details of current study types.
Details of specialty of the principal investigator
Interest for research data network but lack of resources

\paragraph{Data reuse: CDW are used for monitoring and management}
Initialization for billing
Potential for professional feedbacks
Pharmacovigilance

\paragraph{Data reuse: Strong interest for CDW in the context of care}

Some CDWs develop specific applications that provide new functionalities
compared to care software. Search engines can be used to query all the
hospital's data gathered in the CDW, without data compartmentalization between
different softwares. Dedicated interfaces can then offer a unified view of the
history of a patient's data, with inter-specialty transversality, which is
particularly valuable in internal medicine. These cross-disciplinary search
tools also enable healthcare professionals to conduct rapid searches in all the
texts, for example, to find similar patients \citep{garcelon2017finding}.
%
There is a growing interest in such computational phenotyping tools to support
the development of digital health solutions \citep{wen2023impact}.
%
Uses for prevention, automation of repetitive tasks, and care coordination are
also highlighted. Concrete examples are the automatic sorting of hospital
prescriptions by order of complexity or the setting up of specialized channels
for primary or secondary prevention.

\subsection{A multi-layered technical
  architecture}\label{subsec:cdw:results:architecture} Three layer : Data
preprocessing (acquisition and normalization), storage, exposure Datalab: a
crucial technological brick

\subsection{Too little data quality and too many data
  formats}\label{subsec:cdw:results:data_quality}
\paragraph{Quality tools}
Automatic tooling for acquisition
First development for automatic data checks
\paragraph{Standard format}
No single standard data model,
Omop
eHop
\paragraph{Documentation}
Half of the CDWs have put in place documentation accessible within the organization
No documentation is public

\section{Recommendations: How to consolidate EHRs and expand
  usages}\label{sec:cdw:recommendations}

\subsection{Governance: CDWs are
  infrastructures}\label{subsec:cdw:recommendations:governance}

CDW becomes an essential component of data management in the hospital Resources
specific to the warehouse are rare and only project-based. Should promote
long-middle term teams (eg. inria ?) Multi-layered governance

\subsection{Transparency: Keep the bar
  high}\label{subsec:cdw:recommendations:transparency} Public registration of
comparative observational study protocols for research Patient information stays
limited

\subsection{Data: complex data collection requires a variety of
  expertise}\label{subsec:cdw:recommendations:data} study design : change of
focus from data collection to data preprocessing -> other complementary skills
needed Link with the HIS, lack of standard at the HIS level, lack of sharing of
data schema Data reuse oriented towards primary care is still rare and rarely
supported by appropriate funding.


\subsection{Technical architecture: towards more harmonization and open source ?
}\label{subsec:cdw:recommendations:architecture} Lack of harmonization: focus on
fewer solutions Commercial solutions emerging Case for open source:
transparency, less technological lock-in, mutualization, favor modularity, help
build consensus Opportunity for open source solutions Data quality, standard
formats Quality is not sufficiently considered as a relevant scientific topic
itself. Tooling: Link with devops/ automated CI in industrial data science there
is a need for open-source publication of research code to ensure quality
retrospective research

\subsection{Data quality an document: more incentives needed}\label{subsec:cdw:recommendations:quality}

Quality is not sufficiently considered as a relevant scientific topic itself.
However, it is the backbone of all research done within a CDW. In order to
improve the quality of the data with respect to research uses, it is necessary
to conduct continuous studies dedicated to this topic [52,54–56]. These studies
should contribute to a reflection on methodologies and standard tools for data
quality, such as those developed by the OHDSI research network [41].

Finally, there is a need for open-source publication of research code to ensure
quality retrospective research [55,57]. Recent research in data analysis has
shown that innumerable biases can lurk in training data sets [58,59]. Open
publication of data schemas is considered an indispensable prerequisite for all
data science and artificial intelligence uses [58]. Inspired by data set cards
  [58] and data set publication guides, it would be interesting to define a
standard CDW card documenting the main data flows.
\section{Conclusion}\label{sec:cdw:conclusion}

The French CDW ecosystem is beginning to take shape
The priority is the creation and perpetuation of multidisciplinary warehouse teams
Constitution of a multilevel collaboration network is another priority.
Common data model should be encouraged
The question of expanding the scope of the data beyond the purely hospital domain must be asked.


\chapter{Exploring a complexity gradient in representation and predictive algorithms for EHRs}\label{chapter:predictive_models}
\section{Abstract}\label{sec:predictive_models:abstract}

\section{Motivation}\label{sec:predictive_models:motivation}

\subsection{Why such interest in predictive models from EHRs}\label{subsec:predictive_models:importance}

\paragraph{Interest in the clinic}

One of the earliest risk score was designed to predict Coronary heart disease
risk by fitting a cox model using seven features on 5300 patients: age, cholesterol, systolic
blood pressure, hematocrit, ECG status, smoking at intake, and relative body
weight \citep{brand1976multivariate}.

\textit{Being able to predict key outcomes could, theoretically, make the use of
  hospital palliative care resources more efficient and precise} \citep{topol2019high}

Prevention : \textit{Early detection and appropriate treatment of sepsis have
  been associated with a significant mortality benefit in hospitalized patients}
\citep{wong2021external}

Alert systems \citep{yu2018artificial}.

\paragraph{Long term prevention}

\paragraph{Piloting}

\paragraph{Exploring french CDWs:} 23 \% of studies, just after population definitions).

\subsection{EHRs data is complex}\label{subsec:predictive_models:complex_data}
How ? time, high cardinality, multi-modality
In the literature
Within french CDW(link with previous chapter)

\subsection{Low prevalence and local practice}\label{subsec:predictive_models:low_prevalence}

What are the most impactful algorithms for predictive tasks from structured EHR ?
A wealth of methods, but a lack of insights on the advantages and inconvenience for specific problems and resources.

\subsection{What makes a healthcare predictive model useful?}\label{subsec:predictive_models:useful}

We can share it
Performances
Insertion in the care workflow


\section{Predictive pipelines}\label{sec:predictive_models:pipelines}

Demographic features
Count encoding with event features
Static Embeddings of event features
Transformer based
\section{Empirical Study}\label{sec:predictive_models:empirical_study}
\subsection{Evaluation pipeline}\label{subsec:predictive_models:evaluation_pipeline}
\subsection{Task definitions}\label{subsec:predictive_models:task_definitions}

\paragraph{Length Of Stay}
\paragraph{Diagnosis Prediction}
\paragraph{Major Adverse Cardiovascular Events}
\subsection{Results: performance-sample trade-offs}

\section{Conclusion}\label{sec:predictive_models:conclusion}

\chapter{Prediction is not all we need: Causal analysis of EHRs to ground decision making}\label{chapter:causal_tuto}
\section{Abstract}\label{sec:causal_tuto:abstract}

\section{Motivation : Healthcare is concerned with decision making, not mere prediction}\label{sec:causal_tuto:motivation}

\subsection{Predictive medicine currently suffers from biases}\label{subsec:causal_tuto:predictive_medicine_biases}
(shortcuts, population shifts). Racial, gender and under-served population biases raise concern on fairness.
\subsection{A key ingredient to ground data-driven decision making is causal thinking}

\subsection{The need for synthetic materials for practitioners}\label{subsec:causal_tuto:synthetic_materials}

The relevant concepts for causal inference are scattered in different literatures. A dedicated exposition to time-varying data in EHRs would help practitioners and data scientists that study them.
\subsection{Motivating example}\label{subsec:causal_tuto:motivating_example}


\section{Step-by-step framework for robust decision making from EHR data}\label{sec:causal_tuto:framework}

\subsection{Robust study design to avoid biases: Framing the question}
PICOT
Selection Bias
Immortal time bias

\subsection{Is the dataset sufficient to inform on the intervention: identification}
Causal graph Computing the causal effect of interest: Estimation Confounders
extractions Confounders aggregation Causal estimators Nuisance estimators
\subsection{Assessing the robustness of the hypothesis: Vibration or sensitivity
  analysis}\label{subsec:causal_tuto:vibration_analysis}

Sensitivity vs robustness

\subsection{Heterogeneity of treatment}
Interest of HTE
How to do HTE ? Final regression analysis.

\section{Application on MIMIC-IV}\label{sec:causal_tuto:application}

Emulated trial: Effect of albumin in combination with crystalloids compared to crystalloids alone on 28-day mortality in sepsis patients
Choice of the trial
Known effects

\subsection{Framing the question}\label{subsec:causal_tuto:framing_mimic}
\subsection{Identification}
Covariates and dag
\subsection{Estimation}
Confounders extractions
Confounders aggregation
Causal estimators
Nuisance estimators
\subsection{Vibration analysis}
Varying estimation choices:
Varying inclusion criteria: illustration of immortal time bias
\subsection{Heterogeneity of treatment}

\subsection{Discussion}\label{subsec:causal_tuto:discussion}

\chapter{How to select predictive models for causal inference?}\label{chapter:causal_model_selection}
\section{Abstract}\label{sec:causal_model_selection:abstract}

\section{Motivation}\label{sec:causal_model_selection:motivation}

Extending prediction to prescription requires causal model selection
Illustration: the best predictor may not estimate best causal effects
Prior work: model selection for outcome modeling (g-computation)
\section{Methods}\label{sec:causal_model_selection:methods}


\subsection{Notations}\label{subsec:causal_model_selection:notations}

\subsection{Model-selection risks, oracle and feasible}\label{subsec:causal_model_selection:risks}
\subsection{Causal model selection}\label{subsec:causal_model_selection:causal_model_selection}

The $\tau\mathrm{-risk}$ : an oracle error risk
Feasible error risks
Estimation and model selection procedure
\section{Theory: Links between feasible and oracle risks}\label{sec:causal_model_selection:theory}

\subsection{Upper bound of $\tau\mathrm{-risk}$ with
  $\mu\mathrm{-risk}_{IPW}$}\label{subsec:causal_model_selection:upper_bound}

\subsection{Reformulation of the R-risk as reweighted
  $\tau\mathrm{-risk}$}\label{subsec:causal_model_selection:r_risk_reformulation}

\subsection{Interesting special
  cases}\label{subsec:causal_model_selection:special_cases}

\section{Empirical Study}\label{sec:causal_model_selection:empirical_study}

\subsection{Caussim: Extensive simulation settings}\label{subsec:causal_model_selection:caussim}
\paragraph{Data Generation Process}

Family of candidate estimators

\subsection{Semi-simulated datasets}\label{subsec:causal_model_selection:semi_simulated}
\paragraph{Datasets}
\paragraph{Family of candidate estimators}
\paragraph{Nuisance estimators}

\subsection{Measuring overlap between treated and non treated}\label{subsec:causal_model_selection:overlap}

\subsection{Empirical results: factors driving good model selection across
  datasets}\label{subsec:causal_model_selection:empirical_results}


The $R\mathrm{-risk}$ is the best metric

Model selection is harder in settings of low population overlap

Nuisances can be estimated on the same data as outcome
models

Use 90\% of the data to estimate outcome models, 10\% to select them

\section{Discussion and conclusion}\label{sec:causal_model_selection:discussion}


\chapter{Conclusion}\label{chapter:conclusion}

%Predictive medecine is hard

Modern healthcare burdens and costs are driven by chronic diseases where death
is not the only outcome of interest. Focus on smaller rewards, on which
experiments are easier to conduct and where error is possible: it would allow
better learning for decision making since we can repeat more experiments (limit:
signals could be highly delayed).

Text is pervasive, we are using it to communicate and to log most of our
information. We can rely pretraining models outside the healthcare domain. We
should leverage it more in the context of care (limit: temporality is hard to
capture but is a key aspect of causal inference).

The unreasonable effectiveness of healthcare data is yet out of reach due to
hugely difficult transfer of models or administrative barriers to access data
(due to multiplicity of the involved actors). This forces us to rely on
efficient techniques that make the best of medium-sized data or rely on sharable
sources of knowledge (aggregated statistics, federated learning approaches,
ontologies, ...).


\printbibliography

\appendix

\clearpage

\section{Causal Diagrams}

\end{document}
